
\section{EIS: using \texttt{glimmer-tests}}
\texttt{glimmer-tests} provides more example configurations, that include both the EISMINT
and EIS climate drivers. If you have not already done so, download
\texttt{glimmer-tests} via the nescforge page or CVS, and do the usual
\begin{verbatim}
    ./configure -with-glimmer-prefix=/path/to/GLIMMER/installation
    (e.g. /usr/local/GLIMMER)
\end{verbatim}
(if you updated GLIMMER via CVS, you need to do \texttt{./bootstrap} first.)\\
\texttt{glimmer-tests} is not (yet) a test suite, but will exemplarily show
what GLIMMER can do (see the \texttt{glimmer-tests} README file for detailed
information on the tests).

Basically \texttt{glimmer-tests} runs GLIMMER using the EISMINT 1 and 2 (and 3)
climate driver (fixed and moving margin type ice sheets with no external mass
balance forcing), as well as the Edinburgh Ice Sheet (EIS) climate driver,
using mass balance parameterisation via ELA and temperature forcing. There are
a couple of other tests running besides this, e.g. some benchmarks. If you want
to run all the examples, simply do a \texttt{make} in the
\texttt{glimmer-tests} directory, but be aware that running all tests will take a good
12+ hours on a single CPU 3 GHZ machine. If you're too impatient for this,
simply do a \texttt{make} in one of the subdirectories, e.g. EISMINT1 and GLIDE
will be launched using the EISMINT climate driver, which should deliver you a
number of netcdf files with the model results, eg. \texttt{e1.fm.1.nc}
containing the EISMINT1 fixed margin results 1, etc. Again, to visualise the
results use a viewer like ncview.

If you want a more sophisticated results, try \texttt{make} in the \texttt{eis}
directory, which will repeat the results of \cite{Hagdorn2003} reconstructing
the Fennoscandian ice sheet during the last glacial maximum, using the EIS
driver.

\subsection{A short introduction to the EIS driver parameterisation}

Again, check the config file \texttt{fenscan.config} to see the basic
parameters for this model run. Have a look at the \texttt{mb2.data} (mass
balance forcing via ELA), \texttt{temp-exp.model} (exponential type temperature
forcing) and \texttt{specmap.data} (sealevel change) data files and compare
them to the EIS driver documentation (section \ref{driver:eis}) to get an idea of how things are done.

The first column in every data file is the model time at which the new
parameter values are applied. For the temperature model, the records in the
\texttt{temp-exp.model} file
\begin{verbatim}
...
 -97000.000000  -17.858964  23.158964   -0.051329
 -96000.000000  -20.074036  24.674036   -0.051329
...
\end{verbatim}

correspond to the timesteps -97000 and -96000 (first column - model usually
ends at time 0) where the parameters a0 (2nd column), a1 (3rd column) and a2
(last column) of the exponential temperature model $T(t)=a_0+a_1\exp\left(a_2(\lambda-\lambda_0)\right)$
(page \pageref{driver:eis})
are updated to reflect an approximate change in temperature of -2 degrees
Celsius.

For EIS, the mass balance is parameterised via the ELA, according to
$$z_{\text{ELA}} = a + b\lambda + c\lambda^2 + \Delta z_{\text{ELA}},$$
given the parameters in the according config file section:
\begin{verbatim}
...
[EIS ELA]
ela_file = mb2.data
bmax_mar = 4.
ela_a = 14430.069930
ela_b =-371.765734
ela_c = 2.534965
...
\end{verbatim}
Factors a, b and c are specified together with the maximum mass balance of 4. The
latitude $\lambda$ in degrees North is read from the input topography grid. In
order to do the ELA forcing over time, the parameter $\Delta z_{ELA}$ is varied
over time using the ela file \texttt{mb2.data}:
\begin{verbatim}
...
 -109000 225
 -105000 350
...
\end{verbatim}

Similar to the temperature forcing, $\Delta z_{ELA}$ (column 2) is changed at
timestep -10900 (column 1), to reflect an ELA 225m above the altitude value
calculated using the factors a, b, c and the latitude $\lambda$. At timestep
-10500, ELA is rising to 350m above the calculated value.

Where a globally changing $\Delta z_{ELA}$ is insufficient to reflect
disparities in ELA, there are two options to fine tune ELA behaviour. First,
continentality can be used to introduce a dependency of mass balance with
distance to oceans. The according settings are supplied using the \texttt{[EIS CONY]}
section of the config file (see section \ref{driver:eis}).
In short, an index is calculated for every grid cell reflecting 
the ratio of below sealevel cells to land cells within a certain \texttt{range} (defaults to 600km).
Maximum mass balance values are then scaled between the values given in the \texttt{[EIS ELA]} section 
for \texttt{bmax\_mar} (marine conditions, all cells within \texttt{range} are below sea level) and
\texttt{bmax\_cont} (continenal conditions, all cells within \texttt{range} are above sea level).
Alternatively, continentality values between 0 and 1 can be input using a file. Set the according
flag \texttt{file} to 1 and specify the file containing the cony data using a \texttt{[CF input]} section
in the configuration file (see example for ELA file below).
.

In case a more detailed spatial distribution of ELA altitudes is needed, e.g. 
to reflect special orographic effects, a map of $\Delta z_{ELA}$ can be input
to the model using a netcdf file, containing a variable 'ela' on a grid
the same size and coordinates as the input topography grid the model is running
on. This ela file is coupled using a \texttt{[CF input]} section in the
configuration file
\begin{verbatim}
[CF input]
name: ela_1k.nc
\end{verbatim}
resulting in a spatial distribution of $\Delta z_{ELA}$ being applied to the
model. The variation of ELA over time using a global $\Delta z_{ELA}$ is still
applied on top of this ELA forcing file.

\textcolor[rgb]{1.00,0.00,0.00}{\emph{Note: (Maybe an example
containing an ELA forcing file should be added to GLIMMER test/examples?)}}
\\
Sealevel changes are forced upon the model in an according way using the
\texttt{specmap.data} file.
