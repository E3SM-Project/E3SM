\section{GLIDE}
GLIDE is the actual ice sheet model. GLIDE comprises three procedures which initialise the model, perform a single time step and finalise the model. The GLIDE configuration file is described in Section \ref{ug.sec.config}. The GLIDE API is described in Appendix \ref{ug.sec.glide_api}. The simple example driver explains how to write a simple climate driver for GLIDE. Download the example from the GLIMMER website or from CVS:
\begin{verbatim}
cvs -d:pserver:anonymous@forge.nesc.ac.uk:/cvsroot/glimmer login
cvs -z3 -d:pserver:anonymous@forge.nesc.ac.uk:/cvsroot/glimmer co glimmer-example
\end{verbatim}

\subsection{Configuration}\label{ug.sec.config}
The format of the configuration files is similar to Windows \texttt{.ini} files and contains sections. Each section contains key, values pairs.
\begin{itemize}
\item Empty lines, or lines starting with a \texttt{\#}, \texttt{;} or \texttt{!} are ignored.
\item A new section starts with the the section name enclose with square brackets, e.g. \texttt{[grid]}.
\item Keys are separated from their associated values by a \texttt{=} or \texttt{:}.
\end{itemize}
Sections and keys are case sensitive and may contain white space. However, the configuration parser is very simple and thus the number of spaces within a key or section name also matters. Sensible defaults are used when a specific key is not found.

For consistency, options for both the shallow-ice and higher-order dynamical cores (dycore) are discussed. Currently, only the shallow ice dycore is scientifically supported. The higher-order dycore will be supported as part of planned future releases of Glimmer CISM. Configuration number options with a * after them are specific to the higher-order dycore.

\begin{center}
  \tablefirsthead{%
    \hline
  }
  \tablehead{%
    \hline
    \multicolumn{2}{|l|}{\emph{\small continued from previous page}}\\
    \hline
  }
  \tabletail{%
    \hline
    \multicolumn{2}{|r|}{\emph{\small continued on next page}}\\
    \hline}
  \tablelasttail{\hline}
  \begin{supertabular*}{\textwidth}{@{\extracolsep{\fill}}|l|p{10cm}|}
%%%% GRID
    \hline
    \multicolumn{2}{|l|}{\texttt{[grid]}}\\
    \hline
    \multicolumn{2}{|p{0.97\textwidth}|}{Define model grid. Maybe we should make this optional and read grid specifications from input netCDF file (if present). Certainly, the input netCDF files should be checked (but presently are not) if grid specifications are compatible.}\\
    \hline
    \texttt{ewn} & (integer) number of nodes in $x$--direction\\
    \texttt{nsn} & (integer) number of nodes in $y$--direction\\
    \texttt{upn} & (integer) number of nodes in $z$--direction\\
    \texttt{dew} & (real) node spacing in $x$--direction (m)\\
    \texttt{dns} & (real) node spacing in $y$--direction (m)\\
%    \texttt{sigma\_file} & (string) Name of file containing $\sigma$ coordinates. Alternatively, the sigma levels may be specified using the \texttt{[sigma]} section decribed below. If no sigma coordinates are specified explicitly, they are calculated based on the value of \texttt{sigma\_builtin} \\
    \texttt{sigma} &
%    \begin{tabular}[t]{cp{\linewidth}}
%      \multicolumn{2}{p{0.72\textwidth}}{If sigma coordinates are not specified in this configuration file or using the \texttt{sigma\_file} option, this specifies how to compute the sigma coordinates.} \\
      method for specifying sigma coordinates:  \\ &
    \begin{tabular}[t]{cl}
      {\bf 0} & Use Glimmer's default spacing \\[0.05in] 
        & $\sigma_i=\frac{1-(x_i+1)^{-n}}{1-2^{-n}}\quad\mbox{with}\quad x_i=\frac{\sigma_i-1}{\sigma_n-1}, n=2.$ \\[0.05in]
      1 & use sigma coordinates defined in external file (named sigma\_file) \\
      2 & use sigma coordinates given in configuration file \\
      3 & use evenly spaced sigma levels (required by Glam 1st-order dycore) \\
      4 & use Pattyn sigma levels \\
    \end{tabular}\\
    \hline
  
%%%% SIGMA
    \multicolumn{2}{|l|}{\texttt{[sigma]}}\\
    \hline
    \multicolumn{2}{|p{0.95\textwidth}|}{Define the sigma levels used in the vertical discretization (option 2 above). This is an alternative to using a separate file (specified in section \texttt{[grid]} above). If neither is used, the levels are calculated as described above.} \\
    \hline
    \texttt{sigma\_levels} & (real) list of sigma levels, in ascending order, separated by spaces. These run between 0.0 and 1.0 \\
    \hline
%%%% TIME
    \hline
    \multicolumn{2}{|l|}{\texttt{[time]}}\\
    \hline
    \multicolumn{2}{|p{0.95\textwidth}|}{Configure time steps, etc. Update intervals should probably become absolute values rather than related to the main time step when we introduce variable time steps.}\\
    \hline
    \texttt{tstart} & (real) Start time of the model in years\\
    \texttt{tend} & (real) End time of the model in years\\
    \texttt{dt} & (real) size of time step in years\\
    \texttt{ntem} & (real) time step multiplier setting the ice temperature update interval\\
    \texttt{nvel} & (real) time step multiplier setting the velocity update interval\\
    \texttt{dt\_diag} & (real) writing diagnostic variables to log file every dt\_diag yrs\\
    \texttt{idiag} & (int) x direction grid index for diagnostic variables to be written to log file\\
    \texttt{jdiag} & (int) y direction grid index for diagnostic variables to be written to log file\\
    \hline
%%%% Options
    \hline
    \multicolumn{2}{|l|}{\texttt{[options]}}\\
    \hline
    \multicolumn{2}{|p{0.95\textwidth}|}{Parameters set in this section determine how various components of the ice sheet model are treated. Defaults are indicated in bold.}\\
    \hline
%% dycore
    \texttt{dycore} & 
    \begin{tabular}[t]{cp{0.85\linewidth}}
      {\bf 0} & Glide (serial, 3d, Shallow-Ice-Approximation dynamical core)\\
      1* & Glam (parallel, 3d, FDM, 1st-order-accurate dynamical core)  \\
      2* & Glissade (parallel, 3d, FEM, 1st-order-accurate dynamical core)  \\
    \end{tabular}\\
%% evolution
    \texttt{evolution} & 
    \begin{tabular}[t]{ll}
      {\bf 0} & pseudo-diffusion (Glide only)\\
      1 & ADI scheme  (Glide only)\\
      2 & diffusion (Glide only)\\
      3* & incremental remapping (Glam/Glissade only) \\
      4* & first-order upwind (Glam/Glissade only) \\
      5* & evolve without changing ice thickness (Glam/Glissade only) \\
    \end{tabular}\\
%% temperature
    \texttt{temperature} & 
    \begin{tabular}[t]{cl}
      0 & isothermal\\
      {\bf 1} & evolve using Glide temperature scheme \\
      2 & held steady at inital value \\
%      3 & evolve using incremental remapping-based scheme \\
    \end{tabular}\\
%% temperature init
    \texttt{temp\_init} & 
    \begin{tabular}[t]{cp{0.85\linewidth}}
      0 & initial temperatures isothermal at 0 deg. C\\
      {\bf 1} & initial column temperatures set to atmos. temperature \\
      2 & initial column temperatures linearly interpolated from atmos. temperature \\
    \end{tabular}\\
%% flow law
    \texttt{flow\_law} &  
    \begin{tabular}[t]{cp{0.85\linewidth}}
      {\bf 0}  & constant (using the value of default\_flwa)\\
      1 & temp. dependent, Patterson and Budd (temp.=-10degC)\\
      2 & temp. dependent, Patterson and Budd (function of variable temp.)\\
    \end{tabular}\\
%% basal water
    \texttt{basal\_water} & 
    \begin{tabular}[t]{cp{0.85\linewidth}}
      {\bf 0} & none \\
      1 & local water balance\\
      2 & Compute the steady-state, routing-based, basal water flux and water layer thickness \\
      3 &  Use a constant basal water layer thickness everywhere, to enforce T=T\_pmp everywhere \\
    \end{tabular}\\
%% basal mass balance
    \texttt{basal\_mass\_balance} & 
    \begin{tabular}[t]{cl}
      {\bf 0} & ignore basal melt rate in mass balance calculation \\
      1 & include basal melt rate in mass balance calculation \\
    \end{tabular}\\
%% slip coefficient
    \texttt{slip\_coeff} & 
    \begin{tabular}[t]{cl}
      {\bf 0} & zero (no sliding) \\
      1 & set to a non--zero constant everywhere\\
      2 & set to constant where basal water (bwat) is nonzero\\
      3 & set to constant where the ice base is melting\\
      4 & set proportional to basal melt rate\\
      5 & Tanh function of basal water (bwat)\\
    \end{tabular}\\
%% marine margin
    \texttt{marine\_margin} & 
    \begin{tabular}[t]{cp{0.85\linewidth}}
      0 & ignore marine margin\\
      {\bf 1} & Set thickness to zero if floating\\
      2 & Lose fraction of ice when edge cell\\
      3 & Set thickness to zero if relaxed bedrock is below a given depth (variable "mlimit" in glide\_types)\\
      4 & Set thickness to zero if present-day bedrock is below a given depth (variable "mlimit" in glide\_types)\\
    \end{tabular}\\
%% vertical integration
    \texttt{vertical\_integration} & 
    \begin{tabular}[t]{cp{0.85\linewidth}}
      {\bf 0} & standard integration\\
      1 & constrained so that upper boundary kinematic velocity is obeyed\\
    \end{tabular}\\
%% geothermal heat flux
    \texttt{gthf} &  
    \begin{tabular}[t]{cp{0.85\linewidth}}
      {\bf 0} & prescribed, uniform geothermal heat flux \\
      1 & read 2d geothermal heat flux field from input file \\
      2 & calculate geothermal heat flux using 3d diffusion model \\
    \end{tabular}\\
%% isostasy
    \texttt{isostasy} &  
    \begin{tabular}[t]{cp{0.85\linewidth}}
      {\bf 0} & no isostatic adjustment \\
      1 & compute isostatic adjustment using lithosphere / asthenosphere model \\
    \end{tabular}\\
    %% topography
    \texttt{topo\_is\_relaxed} &  
    \begin{tabular}[t]{cp{0.85\linewidth}}
      {\bf 0} & relaxed topography is read from a separate variable\\
      1 & first time slice of input topography is assumed to be relaxed\\
      2 & first time slice of input topography is assumed to be in isostatic
      equilibrium with ice thickness. \\
    \end{tabular}\\
%% periodicity							
    \texttt{periodic\_ew} & 
    \begin{tabular}[t]{cp{0.85\linewidth}}
      {\bf 0} & switched off\\
      1 & periodic lateral EW boundary conditions (Glide SIA dycore only) \\
    \end{tabular}\\
%% hotstart
    \texttt{hotstart} &
    \begin{tabular}[t]{cp{0.85\linewidth}}
      {\bf 0} & normal start (initial guesses taken from input file or, if absent, using default options)\\
      1 & restart model using input from previous run (do not calc. temp., rate factor, or vel. fields )\\
    \end{tabular}\\
    \hline
    \texttt{ioparams} & (string) name of file containing netCDF I/O configuration. The main configuration file is searched for I/O related sections if no file name is given (default).\\
    
 %%%% Higher-order options
    \hline
    \multicolumn{2}{|l|}{\texttt{[ho\_options]}}\\
    \hline
    \multicolumn{2}{|p{0.95\textwidth}|}{Parameters set in this section determine how various components of the higher-order extensions to the ice sheet model are treated. Defaults are indicated in bold. As noted above, higher-order model options are currently NOT scientifically supported.}\\
    \hline
%% which_ho_nonlinear
    \texttt{which\_ho\_nonlinear} & 
    \begin{tabular}[t]{cp{0.85\linewidth}}
      {\bf 0} & treat nonlinearity in momentum balance using Picard iteration \\
      1 & treat nonlinearity in momentum balance using Jacobian-Free Newton-Krylov iteration  \\
    \end{tabular}\\     
%% which_ho_sparse
    \texttt{which\_ho\_sparse} & 
    \begin{tabular}[t]{cp{0.85\linewidth}}
      {\bf 0} & Solve sparse linear system with incomplete LU-preconditioned biconjugate gradient method\\
      1 & Solve sparse linear system with incomplete LU-preconditioned GMRES method\\
      2 & Solve sparse linear system with incomplete LU-preconditioned conjugate gradient method\\
      3 & Solve sparse linear system with conjugate gradient method (parallel code only)\\
      4 & Solve sparse linear system using Trilinos, incomplete LU-preconditioned GMRES method (Trilinos compatible build only)\\
    \end{tabular}\\     
%% which_ho_efvs
    \texttt{which\_ho\_efvs} & 
    \begin{tabular}[t]{cp{0.85\linewidth}}
      0 & use a constant value for the effective viscosity \\
      1 & use a multiple of the flow law rate factor for the effective viscosity \\
      {\bf 2} & use the effective strain rate for the calc. of the effective viscosity (i.e., full nonlinear treatment) \\
    \end{tabular}\\  
%% which_disp
    \texttt{which\_disp} & 
    \begin{tabular}[t]{cp{0.85\linewidth}}
      {\bf 0} & calculate dissipation in temperature equation assuming SIA ice dynamics \\
      1* & calculate dissipation in temperature equation assuming 1st-order ice dynamics \\
    \end{tabular}\\    
%% which_ho_babc
    \texttt{which\_ho\_babc} & 
        Implementation of basal boundary condition in higher-order dycore \\ &
    \begin{tabular}[t]{cp{0.85\linewidth}}
      0 & constant value of 10 Pa yr / m. Useful for debugging\\
      1 & specify some simple pattern (hardcoded). Useful for debugging\\
      2 & treat `betasquared' as yield stress (in Pa) to simulate sliding 
          over a plastic subglacial till (using Picard iteration)\\
%%      3 & calculate betasquared as linear (inverse) function of basal water thickness\\     %% this is the NEW version of this option
      3 & calculate betasquared field as required for circular shelf test case\\     				%% this is the OLD version of this option
      {\bf 4} & no slip everywhere in domain (betasquared set to very
          large value)\\
      5 & read value of betasquared in from .nc input file using standard i/o \\
      4 & no slip everywhere in domain (using Dirichlet basal BC)\\
      7 & Treat `betasquared' as yield stress (in Pa) to simulate sliding 
          over a plastic subglacial till (using Newton iteration - under devel.)\\
    \end{tabular}\\  
%% which_ho_resid
    \texttt{which\_ho\_resid} &
     Residual calculation method for the velocity solver in higher-order dycore
     (iterations halt once residual falls below a specified absolute or relative value) \\ &
    \begin{tabular}[t]{cp{0.85\linewidth}}
      0 & use the maximum value of the normalized velocity vector update, defined by 
      $r = \frac{|vel_{k-1} - vel_k|}{vel_k}$ \\
      1 & as in option 0 but omitting the basal velocities from the comparison
          (useful in cases where an approx. no slip basal BC is enforced) \\
      2 & as in option 0 but using the mean rather than the max \\
      {\bf 3} & use the L2 norm of the system residual, defined by $r = Ax - b$ \\
    \end{tabular}\\  
    \hline
%%%% PARAMETERS
    \hline
    \multicolumn{2}{|l|}{\texttt{[parameters]}}\\
    \hline
    \multicolumn{2}{|p{0.95\textwidth}|}{Set various parameters.}\\
    \hline
    \texttt{log\_level} & (integer) set to a value between 0, no messages, and 6, all messages are displayed to stdout. By default messages are only logged to file.\\
    \texttt{ice\_limit} & (real) below this limit ice is only accumulated; ice dynamics are switched on once the ice thickness is above this value.\\
    \texttt{marine\_limit} & (real) all ice is assumed lost once water depths reach this value (for \texttt{marine\_margin}=2 or 4 in \texttt{[options]} above). Note, water depth is negative. \\
    \texttt{calving\_fraction} & (real) fraction of ice lost due to calving (applies to "marine\_margin" option 2). \\
    \texttt{geothermal} & (real) constant geothermal heat flux.\\
    \texttt{flow\_factor} & (real) the flow law is enhanced with this factor \\
    \texttt{hydro\_time} & (real) basal hydrology time constant \\
    \texttt{isos\_time} & (real) isostasy time constant \\
    \texttt{basal\_tract\_const} & constant basal traction parameter. You can load a nc file with a variable called \texttt{soft} if you want a specially variying bed softness parameter. \\
    \texttt{basal\_tract} & (real(5)) basal traction factors. Basal traction is set to $B=\tanh(W)$ where the parameters
      \begin{tabular}{cp{\linewidth}}
       (1) & width of the $\tanh$ curve\\
       (2) & $W$ at midpoint of $\tanh$ curve [m]\\
       (3) & $B$ minimum [ma$^{-1}$Pa$^{-1}$] \\
       (4) & $B$ maximum [ma$^{-1}$Pa$^{-1}$] \\
       (5) & multiplier for marine sediments \\
     \end{tabular}\\
     \texttt{default\_flwa} * & Flow law parameter A to use in isothermal experiments (flow\_law set to 2).  Default value is $10^{-16}$. \\
    \hline
    \multicolumn{2}{|l|}{\texttt{[isostasy]}}\\
    \hline
    \multicolumn{2}{|p{0.95\textwidth}|}{Isostatic adjustment is only enabled if this section is present in the configuration file. The options described control isostasy model.}\\
    \hline
    \texttt{lithosphere} & \begin{tabular}[t]{cp{0.9\linewidth}} 
      {\bf 0} & local lithosphere, equilibrium bedrock depression is found using Archimedes' principle \\
      1 & elastic lithosphere, flexural rigidity is taken into account
    \end{tabular} \\
    \texttt{asthenosphere} & \begin{tabular}[t]{cp{\linewidth}}
      {\bf 0} & fluid mantle, isostatic adjustment happens instantaneously \\
      1 & relaxing mantle, mantle is approximated by a half-space \\
    \end{tabular} \\    
    \texttt{relaxed\_tau} & characteristic time constant of relaxing mantle (default: 4000.a) \\
    \texttt{update} & lithosphere update period (default: 500.a) \\
    \hline
%%%%
    \hline
    \multicolumn{2}{|l|}{\texttt{[projection]}}\\
    \hline
    \multicolumn{2}{|p{0.95\textwidth}|}{Specify map projection. The reader is
    referred to Snyder J.P. (1987) \emph{Map Projections - a working manual.} USGS 
        Professional Paper 1395. }\\
    \hline
    \texttt{type} & This is a string that specifies the projection type
    (\texttt{LAEA}, \texttt{AEA}, \texttt{LCC} or \texttt{STERE}). \\
    \texttt{centre\_longitude} & Central longitude in degrees east \\
    \texttt{centre\_latitude} & Central latitude in degrees north \\
    \texttt{false\_easting} & False easting in meters \\
    \texttt{false\_northing} & False northing in meters \\
    \texttt{standard\_parallel} & Location of standard parallel(s) in degrees
    north. Up to two standard parallels may be specified (depending on the
    projection). \\
    \texttt{scale\_factor} & non-dimensional. Only relevant for the Stereographic projection.  \\
%%%%
%    \hline
%    \multicolumn{2}{|l|}{\texttt{[elastic lithosphere]}}\\
%    \hline
%    \multicolumn{2}{|p{0.95\textwidth}|}{Set up parameters of the elastic lithosphere.}\\
    \hline
    \texttt{flexural\_rigidity} & flexural rigidity of the lithosphere (default: 0.24e25)\\
    \hline
    \hline
    \multicolumn{2}{|l|}{\texttt{[GTHF]}}\\
    \hline
    \multicolumn{2}{|p{0.95\textwidth}|}{Switch on lithospheric temperature and geothermal heat calculation.}\\
    \hline
    \texttt{num\_dim} & can be either \texttt{1} for 1D calculations or 3 for 3D calculations.\\
    \texttt{nlayer} & number of vertical layers (default: 20). \\
    \texttt{surft} & initial surface temperature (default 2$^\circ$C).\\
    \texttt{rock\_base} & depth below sea-level at which geothermal heat gradient is applied (default: -5000m).\\
    \texttt{numt} & number time steps for spinning up GTHF calculations (default: 0).\\
    \texttt{rho} & The density of lithosphere (default: 3300kg m$^{-3}$).\\
    \texttt{shc} & specific heat capcity of lithosphere (default: 1000J kg$^{-1}$ K$^{-1}$).\\
    \texttt{con} & thermal conductivity of lithosphere (3.3 W m$^{-1}$ K$^{-1}$).\\    
    \hline
  \end{supertabular*}
\end{center}

NetCDF I/O can be configured in the main configuration file or in a separate file (see \texttt{ioparams} in the \texttt{[options]} section). Any number of input and output files can be specified. Input files are processed in the same order they occur in the configuration file, thus potentially overwriting priviously loaded fields.

\begin{center}
  \tablefirsthead{%
    \hline
  }
  \tablehead{%
    \hline
    \multicolumn{2}{|l|}{\emph{\small continued from previous page}}\\
    \hline
  }
  \tabletail{%
    \hline
    \multicolumn{2}{|r|}{\emph{\small continued on next page}}\\
    \hline}
  \tablelasttail{\hline}
  \begin{supertabular*}{\textwidth}{@{\extracolsep{\fill}}|l|p{11cm}|}
%%%% defaults
    \hline
    \multicolumn{2}{|l|}{\texttt{[CF default]}}\\
    \hline
    \multicolumn{2}{|p{0.95\textwidth}|}{This section contains metadata describing the experiment. Any of these parameters can be modified in the \texttt{[output]} section. The model automatically attaches a time stamp and the model version to the netCDF output file.}\\
    \hline
    \texttt{title}& Title of the experiment\\
    \texttt{institution} & Institution at which the experiment was run\\
    \texttt{references} & References that might be useful\\
    \texttt{comment} & A comment, further describing the experiment\\
    \hline
%%%% 
    \hline
    \multicolumn{2}{|l|}{\texttt{[CF input]}}\\
    \hline
    \multicolumn{2}{|p{0.95\textwidth}|}{Any number of input files can be specified. They are processed in the order they occur in the configuration file, potentially overriding previously loaded variables.}\\
    \hline
    \texttt{name}& The name of the netCDF file to be read. Typically netCDF files end with \texttt{.nc}.\\
    \texttt{time}& The time slice to be read from the netCDF file. The first time slice is read by default.\\
    \hline
%%%% 
    \hline
    \multicolumn{2}{|l|}{\texttt{[CF output]}}\\
    \hline
    \multicolumn{2}{|p{0.95\textwidth}|}{This section of the netCDF parameter file controls how often selected  variables are written to file.}\\
    \hline
    \texttt{name} & The name of the output netCDF file. Typically netCDF files end with \texttt{.nc}.\\
    \texttt{start} & Start writing to file when this time is reached (default: first time slice).\\
    \texttt{stop} & Stop writin to file when this time is reached (default: last time slice). \\
    \texttt{frequency} & The time interval in years, determining how often selected variables are written to file.\\
    \texttt{xtype} & Set the floating point representation used in netCDF file. \texttt{xtype} can be one of \texttt{real}, \texttt{double} (default: \texttt{real}). \\
    \texttt{variables} & List of variables to be written to file. See Appendix \ref{ug.sec.varlist} for a list of known variables. Names should be separated by at least one space. The variable names are case sensitive. Variable \texttt{hot} selects all variables necessary for a hotstart.\\
    \hline
  \end{supertabular*}
\end{center}
