\section{CONCLUSION - FUTURE DIRECTION} \label{ch:conclusion}

%%%%%%%%%%%%%%%%%%%%%%%%%%%%%%%%%%%%%%%%%%%%%%%%%%%%%%%%%%%%%%%%%%%%%%%%%%%%%%%%
%\subsection{Introduction}\label{sec:chp6intro}

The majority of the work represented here is designed for what could be accomplished in the future.  Even though there is a lot of benefit in what has been done, the importance of this work only becomes viable through extension, refinement, and application.  Therefore the improvements and possibilities are addressed as well as the conclusion of the immediate work accomplished.

%%%%%%%%%%%%%%%%%%%%%%%%%%%%%%%%%%%%%%%%%%%%%%%%%%%%%%%%%%%%%%%%%%%%%%%%%%%%%%%%
\subsection{Lessons Learned}\label{sec:chp6lessons}

Most of the tools within this thesis were new to me upon beginning.  Although PHAML and GLIMMER-CISM took quite a bit of work to control, usually the smaller items occupied the majority of my time.  The smaller items were tools like Autoconf, Automake, Fortran, MPI, OpenGL, F90GL, Doxygen, SVN, and all of the different compiler options, flags, and dependencies for each tool which caused a lot of headaches.  These types of issues are unavoidable and necessary in order to fully understand how everything is working together.  What I've learned from this is simple:  documentation is essential.

Most projects are good at making sure all the APIs and basic usage is addressed which is very helpful when you are working with the software.  If there is little or inadequate documentation on setting up the tools though, the provided literature is useless.  There is no way to anticipate the number of problems that can occur when trying to compile and install new software, and to this end it is understandable that it is often somewhat neglected.  However, weeks were wasted trying to get libraries that simply would never work because there was no documentation explaining that certain aspects had changed and required different hardware dependencies.  To address this more attention has been spent on the dependencies, setup, installation, and compiling of the tools and libraries.

Developing with a library designed for parallel processing and communications also provided many interesting debugging situations.  As anyone who will work with this library will learn, following all the processes at once can be challenging.  Proper debugging practices become essential and fortunately PHAML provides some nice interfaces in order to individually track the slaves.  Being able to isolate the slave's execution with GDB (The GNU Debugger) is invaluable when tracking data transfer between the master and slave processes.  Above all though, good code practices, testing, and debugging are essential.
%Those working with GLIMMER-CISM in the future will dev

Although not the primary focus of the thesis, I spent a lot of time learning about glaciers, modeling, and partial differential equations.  By taking courses and reading numerous papers for other work with GLIMMER-CISM, I have learned a lot about ice-sheets.  This has given me a better understanding of how important and difficult the study of glaciers is, and also helped me appreciate how CISM actually works in approximating ice-flow.  This knowledge has aided with the integration of PHAML and how PHAML works as well.

%------total or what?



%%%%%%%%%%%%%%%%%%%%%%%%%%%%%%%%%%%%%%%%%%%%%%%%%%%%%%%%%%%%%%%%%%%%%%%%%%%%%%%%
\subsection{Future Improvements}\label{sec:chp6improv}

There are many improvements that could be made to the code between the two software components.  Some of the components that can be improved are the mesh generation, the user module, and the example framework.  Hopefully these improvements will be easy to build upon the current code.  The modules were purposefully made to be easily extended.

\subsubsection{Mesh Generation}
As discussed in section \ref{sec:chp3map} the mesh generation currently works by connecting adjacent nodes that have ice, but this also has a few drawbacks which are not apparent immediately and were also discussed.  A better grounding line identifier for the mesh generation algorithm would be very beneficial in order to separate land masses when adjacent.  This is tricky and would require tracing along the grounding line path in order to maintain a tight approximation of the boundary.  This branching algorithm would be extremely useful for complicated glaciers and especially when a much lower resolution is desired over the entire ice-sheet. Along with the mesh generation a formal triangulation check and verification of the mesh boundaries would be beneficial in assuring that the closest approximation is being taken along the boundary lines.

\subsubsection{PDE Callback Broker}
The PHAML PDE callbacks currently work as a broker which I feel is a good compromise of allowing the functionality to be extendible and flexible.  The current structure though requires each subroutine to check the modnum (module number) which is set in the user module and then call the correct function based on this.  When there are a lot of PHAML modules this process will result in a lot of IF/ELSE statements which is not a very elegant design.  Also, given that the modnum must lie within the user module, which can exist only once, only one PHAML module can be run at a time.  This is an unfortunate consequence of the way the user module works.  More variables could be added to the user module to get around that limitation, but this would be more of a hack.  

A way of running more than one PHAML module at a time would be very desirable, but PHAML wasn't really designed with this kind of framework in mind.  There are definitely ways to make it work now, but these would be poor design decisions and so this issue is left for future work based on individual problem needs.

%%%%%%%%%%%%%%%%%%%%%%%%%%%%%%%%%%%%%%%%%%%%%%%%%%%%%%%%%%%%%%%%%%%%%%%%%%%%%%%%
\subsection{Future Possibilities}\label{sec:chp6poss}

With the basic building blocks in place the phaml\_example module can be copied and tweaked to be used for any specific PDE within GLIMMER-CISM.  This means that several modules could be created each solving some part of the overall glacial system and then integrated back into that calculation.  With this system each component would be isolated from the other ones-meaning making changes to one would be safe from affecting other calculations in unforeseen ways.

Another interesting improvement to CISM would be working with the Pattyn grounding line scheme mentioned earlier.\citep{Pattyn2006JGR}  The code is currently implemented within CISM making this modification partially developed already.  The true calculated grounding line could be added as a node to the mesh and that would provide for a more accurate starting boundary for the PDE to be solved.  This method would also allow for a lower resolution input grid while helping to get rid of the mesh generation issues.

Many other improvements with the triangular meshes are also possible.  The mesh is currently generated at each step, but if CISM stored it the mesh could simply be modified at each timestep based on the mask updates.  Another possibility would be to specify the entire grid as the domain and specify the grounding line points as a boundary within the mesh.  This would allow the resolution to increase along that boundary without the need to modify or rebuilt it.


%%%%%%%%%%%%%%%%%%%%%%%%%%%%%%%%%%%%%%%%%%%%%%%%%%%%%%%%%%%%%%%%%%%%%%%%%%%%%%%%
\subsection{Conclusions}\label{sec:chp6conc}
PHAML is an extremely versatile platform for solving partial differential equations.  A library providing an interface to take advantage of that platform opens many new doors for what can be accomplished with GLIMMER-CISM.  With PHAML hosting its own solvers, adaptive methods, parallel architecture, and optional GUI bindings, a host of new abilities become available from the GLIMMER-CISM framework that before would have required a lot of work and other external libraries to accomplish them.  PHAML is a valuable addition to the GLIMMER-CISM model, incorporating extremely powerful tools into an advanced simulation system.

%In Conclusion, I am awesome.

The ability to integrate useful libraries that are already developed greatly increases the productivity and availability of future glacial research.  By using tools that are already mature it becomes unnecessary to ``re-invent the wheel" by developing libraries from scratch.  Only through this type of open collaboration can progress be made within a decent time-frame.  Creating software for scientific research that is accessible and extendible is therefore of the utmost importance.  GLIMMER-CISM strives to accomplish this with an open community of developers and a model which is open and able to be built upon as shown by the work represented here. 
  

GLIMMER-CISM and PHAML are both useful tools with many features.  Even though quite a lot of work has been done with each of them, this has merely scratched the surface of the potential they encompass together.  %The shear number of options they posses is daunting to say the least, and there is little they shouldn't be able to do together.




