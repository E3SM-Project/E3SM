%=======================================================================
% SVN: $Id: ice_change_ncat.tex 5 2005-12-12 17:41:05Z mvr $
%=======================================================================

The number of ice thickness categories affects ice model input files in three
places:
  \begin{itemize}
    \item {\tt \$NCAT} in the run script
    \item The source code module {\bf ice\_model\_size.F90}
    \item The initial condition (restart) file in the input file directory
  \end{itemize}

The number of ice thickness categories is set in 
{\bf \$CASE/Buildconf/cice.buildexe.csh} using the variable 
called {\tt \$NCAT}.  The default value is 5 categories. {\tt \$NCAT} is 
used to determine the CPP variable setting (NCAT) in 
{\bf ice\_model\_size.F90}. {\tt \$RES} is the resolution of the grid, 
100x116 (gx3v7) and 320x384 (gx1v6) for low and medium resolution grids, 
respectively.

\begin{Ventry}{NOTE:}
\item[NOTE]
To use one ice thickness category, the following changes will need to be
made in the namelist: \\

\begin{verbatim}
  , kitd          = 0
  , kstrength     = 0
\end{verbatim}

With these settings, the model will use the delta scheme instead of linear
remapping and a strength parameterization based on open water area and
mean ice thickness.

\end{Ventry}

The information in the initial restart file is dependent on the number of
ice thickness categories and the total number of layers in the ice distribution.
An initial condition file exists only for the default case of 5 ice thickness
categories, with four layers in each category. To create an initial condition
file for a different number of categories or layers, these steps should be 
followed:

\begin{itemize}
  \item Set {\tt \$NCAT} to the desired number of categories in \\
        {\bf \$CASE/Buildconf/cice.buildexe.csh}.
  \item Set the namelist variable {\tt dumpfreq = 'm'} in \\
        {\bf \$CASE/Buildconf/cice.buildnml.csh} \\
        to print out restart files monthly.
  \item Set the namelist variable {\tt restart = .false.} in 
        {\bf \$CASE/Buildconf/cice.buildnml.csh} \\
        to use the initial conditions within the ice model.
  \item Run the model to equilibrium. 
  \item The last restart file can be used as an initial condition file.
  \item Change the name of the last restart file to {\it iced.0001-01-01.\$GRID}.
  \item Copy the file into the input data directory or directly into the 
        the executable directory.
\end{itemize}

Note that the date printed inside the binary restart file will not be the 
same as 0001-01-01.  For coupled runs, {\tt \$BASEDATE} will be the starting o
date and the date inside the file will not be used.  

